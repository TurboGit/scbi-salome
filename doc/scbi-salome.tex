\documentclass[a4paper,12pt,twoside]{article}

\nonstopmode
\newcommand{\timestamp}{6/06/2019}

\usepackage[T1]{fontenc}
\usepackage[utf8x]{inputenc}
\usepackage{pslatex}  % pour une meilleure portabilité des polices...
\usepackage{hyperref} % pour avoir la structure du document dans acroread
\usepackage[english]{babel}
\usepackage{a4}
\usepackage{listings}
\usepackage[usenames,dvipsnames]{color}
\usepackage[all]{xy}
\usepackage{ltablex}
\usepackage{alltt}
\usepackage{ulem}
\usepackage{cite}
\usepackage{varioref,multicol,multirow,xspace,rotating,moreverb}
\usepackage{makeidx}
\usepackage{enumitem}
\usepackage{tabto}
\usepackage{dirtree}

\widowpenalty=400
\clubpenalty=400
\setlength{\parskip}{1ex plus 1ex minus 1ex}

%%%%%%%%%%%%%%%%%%%%%%%%%%%%%%%%%%%%%%%%%%%%%%%%%%%%%%%%%% SETTINGS

\hyphenation{Raven-SPARK}

\lstset{numbers=none,
	numberstyle=\tiny,
	stepnumber=1,
	numbersep=8pt,
	numberfirstline=true,
	frame=leftline,
	language=Bash,
	tabsize=3,
	breaklines=true,
	xleftmargin=1cm,
	emph={label},
	basicstyle=\small\ttfamily
}

% used to add some spacing before a new item (description)
\let\olditem=\item
\def\myitem[#1]{\vspace{10pt}\olditem[#1]}
%\let\oldbegin\begin
%\def\begin{\let\item\olditem\oldbegin}

% always start the description text in a new line
\setlist[description]{%
	style=nextline,
}

%%%%%%%%%%%%%%%%%%%%%%%%%%%%%%%%%%%%%%%%%%%%%%%%%%%%%%%%%% COMMANDS

\def\myitem2[#1]{\vspace{10pt}\olditem[#1]}

\newenvironment{shellcommand}{
	\begin{list}{ % item
			\bfseries\texttt \$
		}{ % body
			\ttfamily
			\setlength{\topsep}{-0.3ex}
			\setlength{\labelwidth}{1in}
			\setlength{\leftmargin}{0.7in}
			\setlength{\labelsep}{0.5ex}
			\setlength{\rightmargin}{0.5in}
			\setlength{\itemsep}{1ex}
			\setlength{\parsep}{0ex}
			\setlength{\listparindent}{0.5in}
		}
	}{
	\end{list}
}

\newlist{shellcmd}{itemize}{1}
\setlist[shellcmd,1]{label=\textbf{\$}, before=\textit, font=\textit }

\newcommand{\code}[1]{\texttt{#1}}
\newcommand{\tbd}[1]{\fbox{{\smallpencil #1 }}}
\renewcommand{\emph}[1]{\textit{#1}}
\newcommand{\seeref}[1]{see section \ref{#1} p.\pageref{#1}}
\newcommand{\file}[1]{'{\path{#1}}'}
\newcommand{\des}[1]{\tabto{4.5cm}\scriptsize\textit{: #1}}
\newcommand{\cmd}[1]{\tabto{1cm}\hspace{0.5cm}\texttt{\textbf{\$} #1}}
\newcommand{\ddash}{-{}-}

% for all itemize, use bullet
\setitemize[0]{font=\bfseries, label=$\bullet$ }

% let section start on a new page
\let\stdsection\section
\renewcommand\section{\newpage\stdsection}

% no indentation for first line
\setlength{\parindent}{0in}

%%%%%%%%%%%%%%%%%%%%%%%%%%%%%%%%%%%%%%%%%%%%%%%%%%%%%%%%%% TITRE

\title{\huge{S C B I \hspace{2mm} Plug-ins} \\ For SALOME}
\author{Pascal Obry}
\date{July 2019}

%%%%%%%%%%%%%%%%%%%%%%%%%%%%%%%%%%%%%%%%%%%%%%%%%%%%%%%%%% TOC
\begin{document}

\maketitle

\tableofcontents

%%%%%%%%%%%%%%%%%%%%%%%%%%%%%%%%%%%%%%%%%%%%%%%%%%%%%%%%%% SECTION

\section{Introduction}

SCBI stands for Setup Config Build Install.

This document is a supplement for SCBI User's Guide and describes its use in the context of the SALOME project only.

For creating a plug-in module script, and general usage of \code{scbi}, please see the SCBI User's Guide.

%%%%%%%%%%%%%%%%%%%%%%%%%%%%%%%%%%%%%%%%%%%%%%%%%%%%%%%%%% SECTION

\section{SALOME}

The SALOME application is complex and aggregating many libraries and other applications like \emph{ParaView} for example. In this document we will describe how to build it and debug it if needed.

\subsection{Modules}
\label{modules}

There is around 40 SCBI plug-in module scripts for building SALOME. Some modules are pre-requisites (and their dependencies) needed by the core SALOME modules. The core SALOME modules themselves are around ten.

Each pre-requisite and each SALOME module is handled in a single separate SCBI plug-in script. One self contained script per pre-requisite or module.

Finally the meta-module \code{salome} aggregates all of them and contains the necessary code to build the final application and installer.

\subsection{Building}

Building the final application may be as simple as calling:

\cmd{scbi \ddash{}deps \ddash{}env=v93 salome}

This will build SALOME version v9.3 in \file{$HOME/build-salome-v93} using build plan \file{.plan-v93}. Both the \code{v93} environment file and plan are provided by default. This version is targetted to GNU/Debian 9.

Not passing option \code{\ddash{env}=v93} will build SALOME master. That is, all modules will be compiled using the latest version of the code from the repository. This is generally a bad idea and won't compile properly as some specific dependencies are needed like for example \code{Qt 5.9} or \code{OmniORB 4.2.2}.

Note that the default environment files and build plans can be found in \file{$HOME/.config/scbi}.

\subsection{Proxy}

Some modules are hosted on EDF servers and some are on GitHub. It may be needed to go through the proxy when cloning a repository. For this an EDF proxy routine is provided in all default environment files.

\begin{lstlisting}
 function proxy-edf()
 {
    export http_proxy=proxypac.edf.fr:3128
    export https_proxy=proxypac.edf.fr:3128
    export no_proxy=localhost,.edf
 }
\end{lstlisting}

This proxy is also defined in the \code{salome} meta-module.

If you are not using the \code{salome} meta-module and not using the default environment make sure this definition is set in your specific environment file. Optionally it may be defined in the working shell via the \file{.bashrc} or any other specific script.

\subsection{Supported features}

The SCBI driver has support for features that can be enabled or disabled (default). See option \code{\ddash{}enable-<feature>} option in SCBI User's Guide.

In SALOME plug-in scripts three features are in use:

\begin{description}
	\item[application] When enabled the final application will be built.
	\item[installer] When enabled an installer (.run) will be created. This imply option \emph{application}.
	\item[cmake-debug] Enable the debug mode in CMAKE commands which is equivalent to passing the option \code{-DSALOME\_CMAKE\_DEBUG=ON} to \code{cmake}.
\end{description}

For example, for creating the final installer the command line is:

\cmd{scbi \ddash{}deps \ddash{}env=v93 \ddash{}enable-installer salome}

%%%%%%%%%%%%%%%%%%%%%%%%%%%%%%%%%%%%%%%%%%%%%%%%%%%%%%%%%% SECTION

\section{Builds plan}

SALOME comes with some build plans described below.

\begin{description}
	\item[v93] To build SALOME v9.3, the build plan contains the following modules:

	Environment \file{.env-v93}:
	\lstinputlisting{"../scripts.d/.env-v93"}

	Plan \file{.plan-v93}:
	\lstinputlisting{"../scripts.d/.plan-v93"}

	Note the use of the \code{ospray} variant for \code{paraview} module.

	\item[v94] A plan to build SALOME version \code{v9.4}. This builds a more recent version of \code{ParaView} (using a SHA1). It is a work in progress at the time of writing this document.

	Environment \file{.env-v94}:
	\lstinputlisting{"../scripts.d/.env-v94"}

	Plan \file{.plan-v94}:
	\lstinputlisting{"../scripts.d/.plan-v94"}

	\item[next] A plan to build \code{next} version of SALOME. This is used to prepare the next release and experiment with some new dependencies. In this script we build \code{ParaView} with the \code{vrpn} plug-in for example.

	\lstinputlisting{"../scripts.d/.plan-next"}

\end{description}

%%%%%%%%%%%%%%%%%%%%%%%%%%%%%%%%%%%%%%%%%%%%%%%%%%%%%%%%%% SECTION

\section{Preparing a new release}

Preparing a new release may require different steps:

\begin{itemize}
	\item Adding a new pre-prequisite. For this a new SCBI plug-in script is needed, see main SCBI documentation.
	\item Raising a version of a module. For this a new build plan may be needed.
\end{itemize}

\subsection{Raising a module version}

To achieve that the simple way is to copy the current build plan and update it accordingly.

For example, we may copy \file{.plan-v93} to \file{.plan-v94} and edit this later to update the versions of the corresponding modules.

Alternatively it is possible to inherit \code{v93} build-plan in new \file{.plan-v94} (as described above for \file{.plan-next}).

\begin{lstlisting}
 @load v93
 ...
\end{lstlisting}

Finally to build the new version, one can just pass this build plan:

\cmd{scbi \ddash{}deps \ddash{}plan=v94 salome}

This is possible as a build plan is fully independent of the plug-in modules. Note that this will work as long as the new version does not introduce new dependencies and/or options as configuration step.

\subsection{Module with new mandatory dependency}

The current module depends on new mandatory dependency, the corresponding module dependencies plug-ins must be created if needed and referenced into the module \emph{depends} hook.

The \emph{config} hook may require some update. Again see the SCBI User's Guide for a full discussion about this.

\subsection{Module with new optional dependency}

The current module depends on new mandatory dependency, the corresponding module dependencies plug-ins must be created if needed.

The new dependency is not mandatory and we want to have a way to enable it or not at build time, a build variant should be introduced. This build variant may just require a \emph{depends} and \emph{config} variant hooks to activate the dependency at config time. This optional dependency is then activated by specifying the variant in the module reference.

A good example is the \code{paraview} module where multiple variants are proposed to build \code{paraview} with or without \code{ospray} and \code{vrpn}.

\subsection{New module}

For a new module, one have to create a new SCBI plug-in script. See the SCBI User's Guide for a full discussion about this.

%%%%%%%%%%%%%%%%%%%%%%%%%%%%%%%%%%%%%%%%%%%%%%%%%%%%%%%%%% SECTION

\section{Debugging}
\label{debugging}

This section is covered into the main SCBI driver documentation. As this is a quite common and important feature for developers we describe it again here with some examples based on SALOME.

To debug an application one need:

\begin{itemize}
	\item To have access to the sources to change them
	\item Setup SCBI to point to those sources
	\item To compile in debug mode
	\item To build the whole application using the module in debug mode
\end{itemize}

The main feature supporting this is directly built-in SCBI and is called user's checkout.

\subsection{User's checkout}
First the user need to clone the repository to debug. This gives access to the sources. It is possible to clone multiple repositories if the fix involves multiple components.

Let's say that we want to debug the \code{GUI}'s salome module.

\begin{shellcommand}
	\item cd \$HOME/dev/git
	\item git clone https://git.forge.pleiade.edf.fr/git/modules-salome.gui
\end{shellcommand}

This will create a check-out of the GUI's SALOME module in \code{\$HOME/dev}.

\subsection{Build plan}

We want to tell the scbi driver to use the user's checkout instead of the sandbox ones. This is achieved using the specific \code{:dev} version of the module.

We copy the current build plan (assuming the work will be based on SALOME v9.3):

\begin{shellcommand}
	\item cd \$HOME/dev
	\item cp \$HOME/.config/scbi/.plan-v93 .plan-debug
	\item vi .plan-debug
\end{shellcommand}

We change the \code{:V9\_3\_BR} on the \code{salome-gui} line by \code{:dev}, this gives:

\begin{lstlisting}
 salome-gui:dev
\end{lstlisting}

\subsection{Environment}

If the checkout is not in the default location it is needed to set \code{SCBI\_GIT\_REPO}. For the record, the default set on SALOME all environment files is:

\begin{shellcommand}
	\item export SCBI\_GIT\_REPO=\$HOME/dev/git
\end{shellcommand}

For recording a non-default environment, we copy the current default environment and add or edit the \code{SCBI\_GIT\_REPO} variable.

\begin{shellcommand}
	\item cd \$HOME/dev
	\item cp \$HOME/.config/scbi/.env .env-debug
	\item vi .env-debug
\end{shellcommand}

Make sure that one line reads:

\begin{lstlisting}
 SCBI_GIT_REPO=$HOME/dev/git
\end{lstlisting}

We can also record the build plan (created above) to use together with this debug environment:

\begin{lstlisting}
 SCBI_PLAN=debug
\end{lstlisting}

Alternatively the build plan can be specified on the command line.

\subsection{Cleaning}

It may be needed to clean the build tree for a given module to ensure a full and clean restart of the configuration step. This is often needed as \code{cmake} or \code{configure} do keep a cache of configuration and may skip some checks.

To do this scbi come with a \ddash{}purge and \ddash{}purge-only options.

\cmd{scbi \ddash{}purge-only salome-gui}

This command will clean the \code{salome-gui} module only. No compilation will be done.

\cmd{scbi \ddash{}env=debug \ddash{}deps \ddash{}purge salome}

This command will restart the compilation of all modules so doing a full re-compilation of the salome application as each dependency will be cleaned and re-built.

\subsection{Compiling}

To compile SALOME with all dependencies with this debug environment, the simple command is:

\cmd{scbi \ddash{}env=debug \ddash{}deps salome}

If the default environment was sufficient (user's checkout in the default location) we have no new environment created, we just pass the build-plan:

\cmd{scbi \ddash{}plan=debug \ddash{}deps salome}

Finally, if the build-plan has not been specified in the environment by setting \code{SCBI\_PLAN} as described above, we can pass it explicitly on the command line:

\cmd{scbi \ddash{}env=debug \ddash{}plan=debug \ddash{}deps salome}

And we can of course mix the environment and different build-plans if needed.

\cmd{scbi \ddash{}env=v93 \ddash{}plan=debug \ddash{}deps salome}

If more than one module must be debugged and/or changed then just apply the same procedure for the other modules.

\printindex

%%%%%%%%%%%%%%%%%%%%%%%%%%%%%%%%%%%%%%%%%%%%%%%%%%%%%%%%%% SECTION

\end{document}

%%% Local Variables:
%%% TeX-open-quote: "« "
%%% TeX-close-quote: " »"
%%% time-stamp-start: "\\\\newcommand{\\\\timestamp}{"
%%% time-stamp-end: "}"
%%% time-stamp-format: "%02d/%02m/%04y"
%%% End:

% LocalWords:  typographiées PDF Eureka
