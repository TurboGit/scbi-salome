\documentclass[a4paper,12pt,twoside]{article}

\nonstopmode
\newcommand{\timestamp}{6/06/2019}

\usepackage[T1]{fontenc}
\usepackage{pslatex}  % pour une meilleure portabilité des polices...
\usepackage{hyperref} % pour avoir la structure du document dans acroread
\usepackage[english]{babel}
\usepackage{a4}
\usepackage{listings}
\usepackage[usenames,dvipsnames]{color}
\usepackage[all]{xy}
\usepackage{ltablex}
\usepackage{alltt}
\usepackage{ulem}
\usepackage{cite}
\usepackage{varioref,multicol,multirow,xspace,rotating,moreverb}
\usepackage{makeidx}
\usepackage{enumitem}
\usepackage{tabto}
\usepackage{dirtree}

\widowpenalty=400
\clubpenalty=400
\setlength{\parskip}{1ex plus 1ex minus 1ex}

%%%%%%%%%%%%%%%%%%%%%%%%%%%%%%%%%%%%%%%%%%%%%%%%%%%%%%%%%% SETTINGS

\hyphenation{Raven-SPARK}

\lstset{numbers=none,
	numberstyle=\tiny,
	stepnumber=1,
	numbersep=8pt,
	numberfirstline=true,
	frame=leftline,
	language=Bash,
	tabsize=3,
	breaklines=true,
	xleftmargin=1cm,
	emph={label},
	basicstyle=\small\ttfamily
}

% used to add some spacing before a new item (description)
\let\olditem=\item
\def\myitem[#1]{\vspace{10pt}\olditem[#1]}
%\let\oldbegin\begin
%\def\begin{\let\item\olditem\oldbegin}

% always start the description text in a new line
\setlist[description]{%
	style=nextline,
}

%%%%%%%%%%%%%%%%%%%%%%%%%%%%%%%%%%%%%%%%%%%%%%%%%%%%%%%%%% COMMANDS

\def\myitem2[#1]{\vspace{10pt}\olditem[#1]}

\newenvironment{shellcommand}{
	\begin{list}{ % item
			\bfseries\texttt \$
		}{ % body
			\ttfamily
			\setlength{\topsep}{-0.3ex}
			\setlength{\labelwidth}{1in}
			\setlength{\leftmargin}{0.7in}
			\setlength{\labelsep}{0.5ex}
			\setlength{\rightmargin}{0.5in}
			\setlength{\itemsep}{1ex}
			\setlength{\parsep}{0ex}
			\setlength{\listparindent}{0.5in}
		}
	}{
	\end{list}
}

\newlist{shellcmd}{itemize}{1}
\setlist[shellcmd,1]{label=\textbf{\$}, before=\textit, font=\textit }

\newcommand{\code}[1]{\texttt{#1}}
\newcommand{\tbd}[1]{\fbox{{\smallpencil #1 }}}
\renewcommand{\emph}[1]{\textit{#1}}
\newcommand{\seeref}[1]{see section \ref{#1} p.\pageref{#1}}
\newcommand{\file}[1]{'{\path{#1}}'}
\newcommand{\des}[1]{\tabto{4.5cm}\scriptsize\textit{: #1}}
\newcommand{\cmd}[1]{\tabto{1cm}\hspace{0.5cm}\texttt{\textbf{\$} #1}}
\newcommand{\cmdc}[1]{\tabto{1cm}\hspace{1.5cm}\texttt{#1}}
\newcommand{\ddash}{-{}-}

% for all itemize, use bullet
\setitemize[0]{font=\bfseries, label=$\bullet$ }

% let section start on a new page
\let\stdsection\section
\renewcommand\section{\newpage\stdsection}

% no indentation for first line
\setlength{\parindent}{0in}

%%%%%%%%%%%%%%%%%%%%%%%%%%%%%%%%%%%%%%%%%%%%%%%%%%%%%%%%%% TITRE

\title{\huge{S C B I \hspace{2mm} Plug-ins} \\ For SALOME}
\author{Pascal Obry}
\date{May 2021}

%%%%%%%%%%%%%%%%%%%%%%%%%%%%%%%%%%%%%%%%%%%%%%%%%%%%%%%%%% TOC
\begin{document}

\maketitle

\tableofcontents

%%%%%%%%%%%%%%%%%%%%%%%%%%%%%%%%%%%%%%%%%%%%%%%%%%%%%%%%%% SECTION

\section{Introduction}

SCBI stands for Setup Config Build Install.

This document is a supplement for SCBI User's Guide and describes its use in the context of the SALOME project only.

It presents the procedure for creating a plug-in module script, and general usage of \code{scbi}. Readers of this manual is expected to have read the SCBI User's Guide first.

%%%%%%%%%%%%%%%%%%%%%%%%%%%%%%%%%%%%%%%%%%%%%%%%%%%%%%%%%% SECTION

\section{SALOME}

The SALOME application is complex and aggregating many libraries and other applications like \emph{ParaView} for example. In this document we will describe how to build it and debug it if needed.

\subsection{Modules}
\label{modules}

There is around 100 SCBI plug-in module scripts for building SALOME. Some modules are pre-requisites (and theirs dependencies) needed by the core SALOME modules. The core SALOME modules themselves are around ten.

Each pre-requisite and each SALOME module is handled in a single separate SCBI plug-in script. One self contained script per pre-requisite or module.

Finally the meta-module \code{s-salome} aggregates all of them and contains the necessary code to build the final application and installer.

\subsection{setup}

Some modules require a PyPi serveur to be setup. This is done using the PYPIURL key in SCBI's store. For example:

\begin{shellcommand}
\item scbi --env=v96 --store PYPIURL \\
 http://nexus.retd.edf.fr/repository/pypi.python.org/simple/
\end{shellcommand}

\subsection{Building}

Building the final application may be as simple as calling:

\cmd{scbi \ddash{}deps \ddash{}env=v96 s-salome}

This will build SALOME version v9.6 in \file{$HOME/dev/build-salome-v96} using build plan \file{.plan-v96}. Both the \code{v96} environment file and plan are provided by default. This version is targetted to GNU/Debian 10.

Not passing option \code{\ddash{env}=v96} will build SALOME master. That is, all modules will be compiled using the latest version of the code from the repository. This is generally a bad idea and won't compile properly as some specific dependencies are needed like for example \code{Qt 5.9} or \code{OmniORB 4.2.2}.

Note that the default environment files and build plans can be found in \file{$HOME/.config/scbi}.

\subsection{Proxy}

Some modules are hosted on EDF servers and some are on GitHub. It may be needed to go through the proxy when cloning a repository. For this an EDF proxy routine is provided in all default environment files.

\begin{lstlisting}
 function proxy-edf()
 {
    export http_proxy=proxypac.edf.fr:3128
    export https_proxy=proxypac.edf.fr:3128
    export no_proxy=localhost,.edf.fr,127.0.0.1
 }
\end{lstlisting}

This proxy is also defined in the \code{s-salome} meta-module.

If you are not using the \code{s-salome} meta-module and not using the default environment make sure this definition is set in your specific environment file. Optionally it may be defined in the working shell via the \file{.bashrc} or any other specific script.

\subsection{Supported features}

The SCBI driver has support for features that can be enabled or disabled (default). See option \code{\ddash{}enable-<feature>} option in SCBI User's Guide.

In SALOME plug-in scripts three features are in use:

\begin{description}
	\item[application] When enabled the final application will be built.

	\item[installer] When enabled an installer (.run) will be created. This imply option \emph{application}.

	\item[cmake-debug] Enable the debug mode in CMAKE commands which is equivalent to passing the option \code{-DSALOME\_CMAKE\_DEBUG=ON} to \code{cmake}.

	\item[xz] Compresses the installer using the xz algorithm. It can reduce the size of the installer by 50% at the price of a slower generation of the installer.
\end{description}

For example, for creating the final installer the command line is:

\cmd{scbi \ddash{}deps \ddash{}env=v96 \ddash{}enable-installer s-salome}

%%%%%%%%%%%%%%%%%%%%%%%%%%%%%%%%%%%%%%%%%%%%%%%%%%%%%%%%%% SECTION

\section{Builds plan}

SALOME comes with some build plans described below.

\begin{description}
	\item[v96] A plan to build SALOME version \code{v9.6}.

	Environment \file{.env-v96}:
	\lstinputlisting{"../scripts.d/.env-v96"}

	Plan \file{.plan-v96}:
	\lstinputlisting{"../scripts.d/.plan-v96"}

	\item[next] A plan to build \code{next} version of SALOME. This is used to prepare the next release and experiment with some new dependencies. In this script we build \code{ParaView} with the \code{vrpn} plug-in for example.

	\lstinputlisting{"../scripts.d/.plan-next"}

	\item[dev] A build plan and environment to build latest SALOME sources.

	Environment \file{.env-dev}:
	\lstinputlisting{"../scripts.d/.env-dev"}

	Plan \file{.plan-dev}:
	\lstinputlisting{"../scripts.d/.plan-dev"}
\end{description}

%%%%%%%%%%%%%%%%%%%%%%%%%%%%%%%%%%%%%%%%%%%%%%%%%%%%%%%%%% SECTION

\section{Preparing a new release}

Preparing a new release may require different steps:

\begin{itemize}
	\item Adding a new pre-prequisite. For this a new SCBI plug-in script is needed, see main SCBI documentation.
	\item Raising a version of a module. For this a new build plan may be needed.
\end{itemize}

\subsection{Raising modules' version}

To achieve that the simple way is to copy the current build plan and update it accordingly.

For example, we may copy \file{.plan-v95} to \file{.plan-v96} and edit this later to update the versions of the corresponding modules.

Alternatively it is possible to inherit \code{v95} build-plan in new \file{.plan-v96} (as described above for \file{.plan-next}).

\begin{lstlisting}
 @load v95
 ...
\end{lstlisting}

Finally to build the new version, one can just pass this build plan:

\cmd{scbi \ddash{}deps \ddash{}plan=v96 s-salome}

This is possible as a build plan is fully independent of the plug-in modules. Note that this will work as long as the new version does not introduce new dependencies and/or options as configuration step.

If for a quick test, a solution is to specify on the command line the module version needed. For example, let's use build-plan for \code{v96} but use a specific \code{paraview} version based on a sha-1:

\cmd{scbi \ddash{}deps \ddash{}safe \ddash{}plan=v96}
\cmdc{paraview/gdal.ospray:2a6cc49e5f s-salome}

Or, as another example, just build paraview with headless support (which is not the default):

\cmd{scbi \ddash{}deps \ddash{}safe \ddash{}plan=v96}
\cmdc{paraview/gdal.ospray.headless:5.8.0 s-salome}

Note that \code{ospray}, \code{gdal} and \code{headless} are all variants that can be mixed as needed.

\subsection{Module with new mandatory dependency}

The current module depends on new mandatory dependencies, the corresponding module dependencies plug-ins must be created if needed and referenced into the module \emph{depends} hook.

The \emph{config} hook may require some update. Again see the SCBI User's Guide for a full discussion about this.

\subsection{Module with new optional dependency}

The current module depends on new optional dependencies, the corresponding module's dependencies plug-ins must be created if needed.

The new dependency is not mandatory and we want to have a way to enable it or not at build time. In this case a build variant should be introduced. This build variant may just require a \emph{depends} and \emph{config} variant hooks to activate the dependency at config time. This optional dependency is then activated by specifying the variant in the module reference.

A good example is the \code{paraview} module where multiple variants are proposed to build \code{paraview} with or without \code{ospray}, \code{gdal}, \code{headless} and \code{vrpn}.

\subsection{New module}

For a new module, one have to create a new SCBI plug-in script. See the SCBI User's Guide for a full discussion about this.

%%%%%%%%%%%%%%%%%%%%%%%%%%%%%%%%%%%%%%%%%%%%%%%%%%%%%%%%%% SECTION

\section{Building modules on top of a binary installation}

The build on top of a binary distribution is said to be an NG (Next Gen) build.

To build some modules on top of a binary installation the specific plug-in \code{s-salome-bin} (see \ref{pg:salbin}) is provided.

Also two variants are introduced:

\begin{description}
	\item[\code{ng}] : (Next Gen) to build on top of a Linux binary installation.
	\item[\code{ngwin}] : to build on top of a Windows binary installation.
\end{description}

And a new environment (\code{ng-dev}) and meta-module (\code{s-salome-ng}) are also provided. Some Windows specific plug-ins are also available,  see \ref{pg:win}).

\subsection{Building SALOME NG (Next Gen)}

To build SALOME on top of a binary archive with the EDF specific modules added:

\cmd{scbi \ddash{}env=ng-dev \ddash{}deps \ddash{}safe \ddash{}update}
\cmdc{\ddash{}enable-application s-salome-ng}

\subsection{Building modules already in binary distribution}

It may be needed to build a module already in the binary distribution to debug it or compile it with different options.

For this to be possible, the module must be providing an \code{ng} variant. If it is not the case, see \ref{migrateng} for the procedure. Also the followings modules have \code{ng} variants and can be used as source of inspiration:

\begin{itemize}
	\item s-adao
	\item s-atomic
	\item s-geom
	\item s-openturns
	\item s-paraview
	\item s-paravis
	\item s-paravisaddons-edf
	\item s-persalys
	\item s-python-modules
	\item s-python3-xlutils
	\item s-salome-gui
	\item s-salome-jobmanager
	\item s-salome-kernel
	\item s-salome-openturns
	\item s-smesh
	\item s-yacs
	\item s-ydefx
        \item s-zeromq
        \item s-gmsh-plugin
        \item s-netgen-plugin
        \item s-blsurf-plugin
        \item s-ghs3d-plugin
        \item s-ghs3dplr-plugin
        \item s-hybrid-plugin
        \item s-hexotic-plugin
        \item s-shaper
\end{itemize}

When a module has an \code{ng} variant, building it on top of the binary archive can be achieved by calling:

\cmd{scbi \ddash{}env=ng-dev \ddash{}deps \ddash{}safe\ddash{}update s-smesh}

For debugging the module it is recommended to use a User's Checkout (see \ref{user:checkout}) and so building using sources in this specific location:

\cmd{scbi \ddash{}env=ng-dev \ddash{}deps \ddash{}safe \ddash{}update s-smesh/ng:dev}

The setup is then simple, the sources can be edited in the checkout tree, the above command issued each time we want to compile. At the end, the developer can just commit the changes directly in the checkout. This makes the procedure safe and do not require copying the sources manually and yet keeping a controlled environment when the configure, build and install step are provided by the plug-in.

Finally note that on a distribution older than \emph{Debian-10} the documentation build must be disabled, use the option \ddash{}enable-no-doc. This is because \code{sphinx-intl} is not provided by the distribution and we do not want to rebuild part of Sphinx.

\subsection{Integrating new NG modules in application}

We have seen above how to build a module on top of the binary distribution. One generally need to integrate this new module into the final application for testing purpose.

The environment variable \code{SCBI\_EXTRA\_NG\_MODULES} can be used to achieve that.

The syntax is \code{<PLUGIN-1>(PATTERN-1)[:<PLUGIN-2>(PATTERN-2)]}

\begin{description}
	\item[PLUGIN-N] The SCBI plug-in name (must have an ng variant) to be handled as extra module.description
	\item[PATTERN-N] A regular expression of the module as found in the binary archive. It is used to remove this from the NG environment by matching the variable name and the variable values (path). This pattern is not mandatory as already known for supported modules.
\end{description}

For example, to build SMESH and PERSALYS on top of binary archive and integrate them into the final application:

\cmd{export SCBI\_EXTRA\_NG\_MODULES="s-persalys:s-smesh"}

And then using the standard command to build SALOME NG:

\cmd{scbi \ddash{}env=ng-dev \ddash{}deps \ddash{}safe \ddash{}update s-salome-ng}

\subsection{Migrating modules in NG mode}
\label{migrateng}

The following sections give information about the support in SCBI which can be used to create a plug-in compatible with the full build mode or on top of a binary archive (the build named NG).

\subsubsection{Plug-in s-salome-bin}
\label{pg:salbin}

This plug-ins must be used as a dependency of the module to build. It setups the necessary environment like the CMake paths, the include and library paths.

\begin{enumerate}
	\item Add \code{s-salome-bin} in module's build-depends hook.
	\item Update the \code{build-depends} and \code{depends} hook to take the new variant (\code{ng} or \code{ngwin}) into account.
\end{enumerate}

Taking the example of the \code{s-atomic} plug-in:

\begin{itemize}
	\item Rename \code{s-atomic-build-depends} hook to \code{s-atomic-common-build-depends}. This is to share the hook with the standard build.

	\item Rename \code{s-atomic-depends} hook to \code{s-atomic-default-depends}. This is to use the dependencies only for the standard variant and not with the \code{ng} one.

	\item Add \code{s-atomic-linux-build-depends} hook:
	\begin{lstlisting}
	function s-atomic-ng-build-depends()
	{
		echo s-salome-bin
	}
	\end{lstlisting}
\end{itemize}

And that's all is needed. The other hooks to configure, build and install the module are shared with the standard build.

If the module contains a more complex setup with already multiple variants for the configure options the setup may be a bit more challenging than for the above \code{atomic} module. Note that in any cases only the \code{depends}, \code{build-depends} and \code{config-options} hooks need attention. At the end the \code{common} plus \code{ng} variants of the above hooks are what will be used for the \code{ng} variant, and the \code{common}, plus \code{default} hooks must be equivalent to previous version for the non \code{ng} build.

\subsubsection{Plug-ins s-windows-kit and s-visual-c}
\label{pg:win}

For the special case of Windows there is furthermore two other plug-ins (\code{s-windows-kit} and \code{s-visual-c}) that a project must depends on to provide access to Visual Studio and the Windows libraries.

Some variables must be set in the SCBI environment before using those plug-ins.

\begin{description}
	\item[s-windows-kit] To use this plug-in as dependency, the following environment variables must be set:
	\begin{description}
		\item[SCBI\_WINDOWS\_VERSION] The current windows version, for example 10.
		\item[SCBI\_WINDOWS\_KIT\_VERSION] The Windows kit version, for example 10.0.19041.0.
		\item[SCBI\_WINDOWS\_KIT] The path to the Windows kit installation, for example \code{C:/Program Files (x86)/Windows Kits/\$SCBI\_WINDOWS\_VERSION}.
	\end{description}

	\item[s-visual-c] To use the Microsoft visual C compiler, the following environment variable must be set:

	\begin{description}
	\item[SCBI\_WINDOWS\_VC] The path to the Visual C compiler, for example \code{C:/Program Files (x86)/Microsoft Visual Studio/2019/Community/VC/Tools/MSVC/14.29.30133}.
	\end{description}

\end{description}

A good example is the \code{s-paravisaddons-edf} plug-in which can be used to build the EDF specific paravis-addons as part of a full build, on top of a SALOME Linux installation or on top of a SALOME Windows installation.

%%%%%%%%%%%%%%%%%%%%%%%%%%%%%%%%%%%%%%%%%%%%%%%%%%%%%%%%%% INFORMATION

\section{Information about modules}
\label{information}

There is many ways to get information for a module and possible dependencies. We call external in the following section the module that are from the operating system.

\subsection{module's direct dependencies}

The direct dependencies for paraview:

\begin{shellcommand}
	\item scbi \ddash{}env=v96 \ddash{}list-depends:direct paraview
\end{shellcommand}

\begin{lstlisting}
openturns:1.14
ospray:1.8.4
\end{lstlisting}

\subsection{module's dependencies}

\begin{shellcommand}
	\item scbi \ddash{}env=v96 \ddash{}list-depends paraview
\end{shellcommand}

\begin{lstlisting}
embree:3.5.2
ispc:v1.12.0
openturns:1.14
ospray:1.8.4
paraview:v5.8.0
s-clang:llvmorg-8.0.1
\end{lstlisting}

\subsection{module's direct dependencies with externals}

\begin{shellcommand}
	\item scbi \ddash{}env=v96 \ddash{}list-depends:direct \ddash{}list-externals paraview
\end{shellcommand}

\begin{lstlisting}
cmake:3.13.4-1
libgdal-dev:2.4.0+dfsg-1+b1
libhdf5-dev:1.10.4+repack-10
libopenmpi-dev:3.1.3-11
libqt5help5:5.11.3-4
libqt5svg5-dev:5.11.3-2
libqt5x11extras5-dev:5.11.3-2
libx11-dev:2:1.6.7-1+deb10u1
libxt-dev:1:1.1.5-1+b3
openturns:1.14
ospray:1.8.4
python3-dev:3.7.3-1
qt5-default:5.11.3+dfsg1-1+deb10u4
qttools5-dev:5.11.3-4
qtxmlpatterns5-dev-tools:5.11.3-2
\end{lstlisting}

\subsection{module's dependencies with externals}

\begin{shellcommand}
	\item scbi \ddash{}env=v96 \ddash{}list-externals paraview
\end{shellcommand}

\begin{lstlisting}
bison:2:3.3.2.dfsg-1
cmake:3.13.4-1
embree:3.5.2
flex:2.6.4-6.2
ispc:v1.12.0
libblas-dev:3.8.0-2
libc6-dev:2.28-10
libcminpack-dev:1.3.6-4
libgdal-dev:2.4.0+dfsg-1+b1
libglfw3-dev:3.2.1-1
libhdf5-dev:1.10.4+repack-10
liblapack-dev:3.8.0-2
libncurses-dev:6.1+20181013-2+deb10u2
libopenmpi-dev:3.1.3-11
libqt5help5:5.11.3-4
libqt5svg5-dev:5.11.3-2
libqt5x11extras5-dev:5.11.3-2
libtbb-dev:2018~U6-4
libx11-dev:2:1.6.7-1+deb10u1
libxt-dev:1:1.1.5-1+b3
m4:1.4.18-2
openturns:1.14
ospray:1.8.4
paraview:v5.8.0
python3:3.7.3-1
python3-dev:3.7.3-1
qt5-default:5.11.3+dfsg1-1+deb10u4
qttools5-dev:5.11.3-4
qtxmlpatterns5-dev-tools:5.11.3-2
s-clang:llvmorg-8.0.1
zlib1g-dev:1:1.2.11.dfsg-1
\end{lstlisting}

We can also get modules without the version information with option \ddash{}list-no-version:

\begin{shellcommand}
	\item scbi \ddash{}env=v96 \ddash{}list-externals \ddash{}list-no-version paraview
\end{shellcommand}

\begin{lstlisting}
bison
cmake
embree
flex
ispc
libblas-dev
libc6-dev
libcminpack-dev
libgdal-dev
libglfw3-dev
libhdf5-dev
liblapack-dev
libncurses-dev
libopenmpi-dev
libqt5help5
libqt5svg5-dev
libqt5x11extras5-dev
libtbb-dev
libx11-dev
libxt-dev
m4
openturns
ospray
paraview
python3
python3-dev
qt5-default
qttools5-dev
qtxmlpatterns5-dev-tools
s-clang
zlib1g-dev
\end{lstlisting}

%%%%%%%%%%%%%%%%%%%%%%%%%%%%%%%%%%%%%%%%%%%%%%%%%%%%%%%%%% SECTION

\section{Debugging}
\label{debugging}

There is two solutions to debug a code:

\begin{itemize}
	\item User's checkout
	\item Running only configuration or build step
	\item SCBI shell
\end{itemize}

Those solutions are discussed below.

\subsection{User's checkout}

This section is covered into the main SCBI driver documentation. As this is a quite common and important feature for developers we describe it again here with some examples based on SALOME.

To debug an application one need:

\begin{itemize}
	\item To have access to the sources to change them
	\item Setup SCBI to point to those sources
	\item To compile in debug mode
	\item To build the whole application using the module in debug mode
\end{itemize}

The main feature supporting this is directly built-in SCBI and is called user's checkout.

\subsubsection{User's checkout}
\label{user:checkout}
First the user need to clone the repository to debug. This gives access to the sources. It is possible to clone multiple repositories if the fix involves multiple components.

Let's say that we want to debug the \code{GUI}'s salome module.

\begin{shellcommand}
	\item cd \$HOME/dev/git
	\item git clone https://gitlab.pleiade.edf.fr/salome/gui
\end{shellcommand}

This will create a check-out of the GUI's SALOME module in \code{\$HOME/dev}.

\subsubsection{Build plan}

We want to tell the scbi driver to use the user's checkout instead of the sandbox ones. This is achieved using the specific \code{:dev} version of the module.

We copy the current build plan (assuming the work will be based on SALOME v9.3):

\begin{shellcommand}
	\item cd \$HOME/dev
	\item cp \$HOME/.config/scbi/.plan-v96 .plan-debug
	\item vi .plan-debug
\end{shellcommand}

We change the \code{:V9\_3\_BR} on the \code{s-salome-gui} line by \code{:dev}, this gives:

\begin{lstlisting}
 s-salome-gui:dev
\end{lstlisting}

Alternatively it is possible to specify the \code{:dev} version directly from the commande line. This may be easier as a quick solution:

\cmd{scbi \ddash{}deps \ddash{}plan=v96 s-salome-gui:dev s-salome}

\subsubsection{Environment}

If the checkout is not in the default location it is needed to set \code{SCBI\_GIT\_REPO}. For the record, the default set on SALOME all environment files is:

\begin{shellcommand}
	\item export SCBI\_GIT\_REPO=\$HOME/dev/git
\end{shellcommand}

For recording a non-default environment, we copy the current default environment and add or edit the \code{SCBI\_GIT\_REPO} variable.

\begin{shellcommand}
	\item cd \$HOME/dev
	\item cp \$HOME/.config/scbi/.env .env-debug
	\item vi .env-debug
\end{shellcommand}

Make sure that one line reads:

\begin{lstlisting}
 SCBI_GIT_REPO=$HOME/dev/git
\end{lstlisting}

We can also record the build plan (created above) to use together with this debug environment:

\begin{lstlisting}
 SCBI_PLAN=debug
\end{lstlisting}

Alternatively the build plan can be specified on the command line.

\subsubsection{Cleaning}

It may be needed to clean the build tree for a given module to ensure a full and clean restart of the configuration step. This is often needed as \code{cmake} or \code{configure} do keep a cache of configuration and may skip some checks.

To do this scbi come with a \ddash{}purge and \ddash{}purge-only options.

\cmd{scbi \ddash{}purge-only s-salome-gui}

This command will clean the \code{s-salome-gui} module only. No compilation will be done.

\cmd{scbi \ddash{}env=debug \ddash{}deps \ddash{}purge s-salome}

This command will restart the compilation of all modules so doing a full re-compilation of the salome application as each dependency will be cleaned and re-built.

\subsubsection{Compiling}

To compile SALOME with all dependencies with this debug environment, the simple command is:

\cmd{scbi \ddash{}env=debug \ddash{}safe \ddash{}deps s-salome}

If the default environment was sufficient (user's checkout in the default location) we have no new environment created, we just pass the build-plan:

\cmd{scbi \ddash{}plan=debug \ddash{}safe \ddash{}deps s-salome}

Finally, if the build-plan has not been specified in the environment by setting \code{SCBI\_PLAN} as described above, we can pass it explicitly on the command line:

\cmd{scbi \ddash{}env=debug \ddash{}safe \ddash{}plan=debug \ddash{}deps s-salome}

And we can of course mix the environment and different build-plans if needed.

\cmd{scbi \ddash{}env=v96 \ddash{}safe \ddash{}plan=debug \ddash{}deps s-salome}

If more than one module must be debugged and/or changed then just apply the same procedure for the other modules.

Note that the \ddash{}safe option ensure that every module for which the context has changed (including the SCBI module's script) will be recompiled.

\subsection{Running configuration or build step}

When debugging it may also be useful to just restart a configuration or build step of a specific module after having modified the sources. Note that changes in the sandbox will be lost as soon as a full SCBI build will be done. So be sure to get the modified sources out of the sandbox when the issue is resolved.

\subsubsection{Configuration only}

To run only the configuration step, and have the output on the console:

\cmd{scbi \ddash{}env=debug \ddash{}log:yes \ddash{}config s-salome-kernel}

\subsubsection{Build only}

To run only the build step, and have the output on the console:

\cmd{scbi \ddash{}env=debug \ddash{}log:yes \ddash{}build s-salome-kernel}

\subsection{Using the SCBI shell}

Another alternative is to go into the sandbox with proper environment and debug in-place the module.

\cmd{scbi \ddash{}env=v96 \ddash{}shell:sandbox paraview}

This command will open a sub-shell into the build directory of paraview in the sandbox. The environment variables are set exactly with the same values as when compiling in the sandbox. The sources can be changed directly in directory whose relative path is \file{../src}. After changing the paraview sources a simple \code{make} will recompile paraview.

Note that it is discouraged to use this method and the preferred solution is to use a user's checkout, \seeref{user:checkout}. This is because in the sandbox the sources will be overwritten when starting a standard compilation. Remember the sandbox is a controlled environment which is fully handled by \code{SCBI}.

\printindex

%%%%%%%%%%%%%%%%%%%%%%%%%%%%%%%%%%%%%%%%%%%%%%%%%%%%%%%%%% SECTION

\end{document}

%%% Local Variables:
%%% TeX-open-quote: "« "
%%% TeX-close-quote: " »"
%%% time-stamp-start: "\\\\newcommand{\\\\timestamp}{"
%%% time-stamp-end: "}"
%%% time-stamp-format: "%02d/%02m/%04y"
%%% End:

% LocalWords:  typographiées PDF Eureka
